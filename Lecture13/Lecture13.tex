\documentclass[
  shownotes,
  xcolor={svgnames},
  hyperref={colorlinks,citecolor=DarkBlue,linkcolor=DarkRed,urlcolor=DarkBlue}
  ]{beamer}
\usepackage{animate}
\usepackage{amsmath}
\usepackage{amsfonts}
\usepackage{amssymb}
\usepackage{pifont}
\usepackage{mathpazo}
%\usepackage{xcolor}
\usepackage{multimedia}
\usepackage{fancybox}
\usepackage[para]{threeparttable}
\usepackage{multirow}
\setcounter{MaxMatrixCols}{30}
\usepackage{subcaption}
\usepackage{graphicx}
\usepackage{lscape}
\usepackage[compatibility=false,font=small]{caption}
\usepackage{booktabs}
\usepackage{ragged2e}
\usepackage{chronosys}
\usepackage{appendixnumberbeamer}
\usepackage{animate}
\setbeamertemplate{caption}[numbered]
\usepackage{color}
%\usepackage{times}
\usepackage{tikz}
\usepackage{comment} %to comment
%% BibTeX settings
\usepackage{natbib}
\bibliographystyle{apalike}
\bibpunct{(}{)}{,}{a}{,}{,}
\setbeamertemplate{bibliography item}{[\theenumiv]}

% Defines columns for bespoke tables
\usepackage{array}
\newcolumntype{L}[1]{>{\raggedright\let\newline\\\arraybackslash\hspace{0pt}}m{#1}}
\newcolumntype{C}[1]{>{\centering\let\newline\\\arraybackslash\hspace{0pt}}m{#1}}
\newcolumntype{R}[1]{>{\raggedleft\let\newline\\\arraybackslash\hspace{0pt}}m{#1}}


\usepackage{xfrac}


\usepackage{multicol}
\setlength{\columnsep}{0.5cm}

% Theme and colors
\usetheme{Boadilla}

% I use steel blue and a custom color palette. This defines it.
\definecolor{andesred}{HTML}{af2433}

% Other options
\providecommand{\U}[1]{\protect\rule{.1in}{.1in}}
\usefonttheme{serif}
\setbeamertemplate{itemize items}[default]
\setbeamertemplate{enumerate items}[square]
\setbeamertemplate{section in toc}[circle]

\makeatletter

\definecolor{mybackground}{HTML}{82CAFA}
\definecolor{myforeground}{HTML}{0000A0}

\setbeamercolor{normal text}{fg=black,bg=white}
\setbeamercolor{alerted text}{fg=red}
\setbeamercolor{example text}{fg=black}

\setbeamercolor{background canvas}{fg=myforeground, bg=white}
\setbeamercolor{background}{fg=myforeground, bg=mybackground}

\setbeamercolor{palette primary}{fg=black, bg=gray!30!white}
\setbeamercolor{palette secondary}{fg=black, bg=gray!20!white}
\setbeamercolor{palette tertiary}{fg=white, bg=andesred}

\setbeamercolor{frametitle}{fg=andesred}
\setbeamercolor{title}{fg=andesred}
\setbeamercolor{block title}{fg=andesred}
\setbeamercolor{itemize item}{fg=andesred}
\setbeamercolor{itemize subitem}{fg=andesred}
\setbeamercolor{itemize subsubitem}{fg=andesred}
\setbeamercolor{enumerate item}{fg=andesred}
\setbeamercolor{item projected}{bg=gray!30!white,fg=andesred}
\setbeamercolor{enumerate subitem}{fg=andesred}
\setbeamercolor{section number projected}{bg=gray!30!white,fg=andesred}
\setbeamercolor{section in toc}{fg=andesred}
\setbeamercolor{caption name}{fg=andesred}
\setbeamercolor{button}{bg=gray!30!white,fg=andesred}


\usepackage{fancyvrb}
\newcommand{\VerbBar}{|}
\newcommand{\VERB}{\Verb[commandchars=\\\{\}]}
\DefineVerbatimEnvironment{Highlighting}{Verbatim}{commandchars=\\\{\}}
% Add ',fontsize=\small' for more characters per line
\usepackage{framed}
\definecolor{shadecolor}{RGB}{248,248,248}
\newenvironment{Shaded}{\begin{snugshade}}{\end{snugshade}}
\newcommand{\AlertTok}[1]{\textcolor[rgb]{0.94,0.16,0.16}{#1}}
\newcommand{\AnnotationTok}[1]{\textcolor[rgb]{0.56,0.35,0.01}{\textbf{\textit{#1}}}}
\newcommand{\AttributeTok}[1]{\textcolor[rgb]{0.77,0.63,0.00}{#1}}
\newcommand{\BaseNTok}[1]{\textcolor[rgb]{0.00,0.00,0.81}{#1}}
\newcommand{\BuiltInTok}[1]{#1}
\newcommand{\CharTok}[1]{\textcolor[rgb]{0.31,0.60,0.02}{#1}}
\newcommand{\CommentTok}[1]{\textcolor[rgb]{0.56,0.35,0.01}{\textit{#1}}}
\newcommand{\CommentVarTok}[1]{\textcolor[rgb]{0.56,0.35,0.01}{\textbf{\textit{#1}}}}
\newcommand{\ConstantTok}[1]{\textcolor[rgb]{0.00,0.00,0.00}{#1}}
\newcommand{\ControlFlowTok}[1]{\textcolor[rgb]{0.13,0.29,0.53}{\textbf{#1}}}
\newcommand{\DataTypeTok}[1]{\textcolor[rgb]{0.13,0.29,0.53}{#1}}
\newcommand{\DecValTok}[1]{\textcolor[rgb]{0.00,0.00,0.81}{#1}}
\newcommand{\DocumentationTok}[1]{\textcolor[rgb]{0.56,0.35,0.01}{\textbf{\textit{#1}}}}
\newcommand{\ErrorTok}[1]{\textcolor[rgb]{0.64,0.00,0.00}{\textbf{#1}}}
\newcommand{\ExtensionTok}[1]{#1}
\newcommand{\FloatTok}[1]{\textcolor[rgb]{0.00,0.00,0.81}{#1}}
\newcommand{\FunctionTok}[1]{\textcolor[rgb]{0.00,0.00,0.00}{#1}}
\newcommand{\ImportTok}[1]{#1}
\newcommand{\InformationTok}[1]{\textcolor[rgb]{0.56,0.35,0.01}{\textbf{\textit{#1}}}}
\newcommand{\KeywordTok}[1]{\textcolor[rgb]{0.13,0.29,0.53}{\textbf{#1}}}
\newcommand{\NormalTok}[1]{#1}
\newcommand{\OperatorTok}[1]{\textcolor[rgb]{0.81,0.36,0.00}{\textbf{#1}}}
\newcommand{\OtherTok}[1]{\textcolor[rgb]{0.56,0.35,0.01}{#1}}
\newcommand{\PreprocessorTok}[1]{\textcolor[rgb]{0.56,0.35,0.01}{\textit{#1}}}
\newcommand{\RegionMarkerTok}[1]{#1}
\newcommand{\SpecialCharTok}[1]{\textcolor[rgb]{0.00,0.00,0.00}{#1}}
\newcommand{\SpecialStringTok}[1]{\textcolor[rgb]{0.31,0.60,0.02}{#1}}
\newcommand{\StringTok}[1]{\textcolor[rgb]{0.31,0.60,0.02}{#1}}
\newcommand{\VariableTok}[1]{\textcolor[rgb]{0.00,0.00,0.00}{#1}}
\newcommand{\VerbatimStringTok}[1]{\textcolor[rgb]{0.31,0.60,0.02}{#1}}
\newcommand{\WarningTok}[1]{\textcolor[rgb]{0.56,0.35,0.01}{\textbf{\textit{#1}}}}
\usepackage{graphicx}
\makeatletter

\definecolor{airforceblue}{rgb}{0.36, 0.54, 0.66}

\usepackage{tikz}
% Tikz settings optimized for causal graphs.
\usetikzlibrary{shapes,decorations,arrows,calc,arrows.meta,fit,positioning}
\tikzset{
    -Latex,auto,node distance =1 cm and 1 cm,semithick,
    state/.style ={ellipse, draw, minimum width = 0.7 cm},
    point/.style = {circle, draw, inner sep=0.04cm,fill,node contents={}},
    bidirected/.style={Latex-Latex,dashed},
    el/.style = {inner sep=2pt, align=left, sloped}
}


\makeatother






%%%%%%%%%%%%%%% BEGINS DOCUMENT %%%%%%%%%%%%%%%%%%

\begin{document}

\title[Lecture 13]{Lecture 13: \\ Spatial Models (Cont.)}
\subtitle{Big Data and Machine Learning for Applied Economics \\ Econ 4676}
\date{\today}

\author[Sarmiento-Barbieri]{Ignacio Sarmiento-Barbieri}
\institute[Uniandes]{Universidad de los Andes}


\begin{frame}[noframenumbering]
\maketitle
\end{frame}

%%%%%%%%%%%%%%%%%%%%%%%%%%%%%%%%%%%

%----------------------------------------------------------------------%
\begin{frame}
\frametitle{Announcements}

\begin{itemize}
  \item Final Project
  \begin{itemize}
    \item First deadline. Sept 25.  Brief zoom hang out, short presentation (5 slides tops). Present idea and basic plan. \textcolor{airforceblue}{soft deadline}
     \medskip
    \item  Second deadline. October 25. Show data. \textcolor{andesred}{hard deadline}
      \medskip
    \item  Final work, December 17. Bonus for ``complete papers'' \textcolor{andesred}{hard deadline}
  \end{itemize}
\bigskip
    \item Prof. Tomás Rodríguez Barraquer will jump in and I'll leave
\end{itemize}

\end{frame}
%----------------------------------------------------------------------%
\begin{frame}
\frametitle{Recap }

  \begin{itemize} 
        \item Closeness
        \medskip
        \item Weights matrix
        \medskip
        \item Examples of weight matrices weights matrix in \texttt{R}
        \medskip
        \item Traditional spatial regressions
        \medskip
        \item Prediction with spatial model 
  \end{itemize}

\end{frame}

%----------------------------------------------------------------------% 

\begin{frame}
\frametitle{Agenda}

\tableofcontents

\end{frame}




%----------------------------------------------------------------------%
\section{Motivation }
%----------------------------------------------------------------------%
\begin{frame}[fragile]
\frametitle{Motivation}


{\it “Everything is related to everything else, but close things are more related than things that are far apart”} (Tobler, 1979).

\bigskip

\begin{itemize}
  
  \item Independence assumption between observation is no longer valid
  \medskip
  \item Attributes of observation $i$  may influence the attributes of observation $j$.
  \medskip
  \item Today we will dive into the estimation of spatial lag models.
  \medskip
  \item Think as a way to model $f(X)$
  \medskip
  \item Spatial dependence introduces a miss specification problem
\end{itemize}



\end{frame}
%----------------------------------------------------------------------%
\section{Spatial Lag Model}
%----------------------------------------------------------------------%
\begin{frame}[fragile]
\frametitle{Spatial Lag Model}

Let's consider the following model:
\bigskip 
\[
y=\lambda Wy+X\beta+u
\]
\medskip
with $|\lambda|<1$, we also assume that $W$ is exogenous

\medskip

If $W$ is row standardized:
\medskip
\begin{itemize}
\item Guarantees $|\lambda|<1$ (Anselin, 1982) 
\item $[0,1]$ Weights
\item $Wy$ Average of neighboring values
\item W is no longer symmetric $\sum_{j}w_{ij}\neq\sum_{i}w_{ji}$ (complicates
computation)
\end{itemize}

\end{frame}
%----------------------------------------------------------------------%
\begin{frame}[fragile]
\frametitle{Spatial Lag Model}
\framesubtitle{Maximum Likelihood Estimator}

Note that we can write 
\bigskip 
\[
(I-\lambda W)y=X\beta+u
\]
\bigskip 
\begin{itemize}
\item We can think this model as a way to correct for loss of information coming from spatial dependence.
\bigskip 
\item $(1-\lambda W)y$ is a spatially filtered dependent variable, i.e.,
the effect of spatial autocorrelation taken out
\end{itemize}

\end{frame}
%----------------------------------------------------------------------%
\begin{frame}[fragile]
\frametitle{Spatial Lag Model}

In this case, endogeneity emerges because the spatially lagged value
of y is correlated with the stochastic disturbance. 

\[
y=(I-\lambda W)^{-1}X\beta+(I-\lambda W)^{-1}u
\]

\[
E((Wy)u')=E(W(I-\lambda W)^{-1}X\beta u'+W(I-\lambda W)^{-1}uu')
\]

\[
=W(I-\lambda W)^{-1}X\beta E(u')+W(I-\lambda W)^{-1}E(uu')
\]

\[
=W(I-\lambda W)^{-1}E(uu')
\]

\[
=\sigma^{2}W(I-\lambda W)^{-1}\neq0
\]

\end{frame}
%----------------------------------------------------------------------%
\subsection{Maximum Likelihood Estimator}
%----------------------------------------------------------------------%
\begin{frame}[fragile]
\frametitle{Spatial Lag Model}
\framesubtitle{Maximum Likelihood Estimator}
\begin{itemize}
\item One solution that emerged in the literature is MLE
\item We need an extra assumption, i.e.,  $u\sim_{iid}N(0,\sigma^{2}I)$. 
\end{itemize}

\[
y=(I-\lambda W)^{-1}X\beta+(I-\lambda W)^{-1}u
\]

note that

\[
E(y)=(I-\lambda W)^{-1}X\beta+(I-\lambda W)^{-1}u
\]

\[
=(I-\lambda W)^{-1}X\beta+(I-\lambda W)^{-1}E(u)
\]

\[
=(I-\lambda W)^{-1}X\beta
\]


\end{frame}
%----------------------------------------------------------------------%
\begin{frame}[fragile]
\frametitle{Spatial Lag Model}
\framesubtitle{Maximum Likelihood Estimator}



\begin{align}
E(yy') & =((I-\lambda W)^{-1}X\beta+(I-\lambda W)^{-1}u)((I-\lambda W)^{-1}X\beta+(I-\lambda W)^{-1}u)' \nonumber \\ 
& = (I-\lambda W)^{-1}X\beta\beta'X'(I-\lambda W')^{-1}+(I-\lambda W)^{-1}u\beta'X'(I-\lambda W')^{-1} \nonumber \\
&+(I-\lambda W)^{-1}X\beta u'(I-\lambda W')^{-1}+(I-\lambda W)^{-1}uu'(I-\lambda W')^{-1}  \nonumber\\
&=(I-\lambda W)^{-1}X\beta\beta'X'(I-\lambda W')^{-1}+(I-\lambda W)^{-1}uu'(I-\lambda W')^{-1}  \nonumber \\
&=(I-\lambda W)^{-1}X\beta\beta'X'(I-\lambda W')^{-1}+(I-\lambda W)^{-1}(I-\lambda W')^{-1}\sigma^{2} \nonumber 
\end{align}

then

\begin{align}
V(y)&=E(yy')-E(y) \nonumber \\ 
&=[(I-\lambda W)'(I-\lambda W)]^{-1}\sigma^{2} \nonumber \\ 
&=\Omega\sigma^{2}
\end{align}


\end{frame}
%----------------------------------------------------------------------%
\begin{frame}[fragile]
\frametitle{Spatial Lag Model}
\framesubtitle{Maximum Likelihood Estimator}


The associated likelihood function is then
\begin{scriptsize}
\begin{equation}
\mathcal{L}\left(\sigma^{2},\lambda,y\right)=\left(\frac{1}{\sqrt{2\pi}}\right)^{n}|\sigma^{2}\Omega|^{-\frac{1}{2}}exp\left\{ -\frac{1}{2\sigma^{2}}(y-(I-\lambda W)^{-1}X\beta)'\Omega^{-1}(y-(I-\lambda W)^{-1}X\beta)\right\}  \nonumber
\end{equation}
\end{scriptsize}


the log likelihood

\begin{scriptsize}
\begin{equation}
l\left(\sigma^{2},\lambda,y\right)=constant-\frac{1}{2}ln|\sigma^{2}\Omega|-\frac{1}{2\sigma^{2}}(y-(I-\lambda W)^{-1}X\beta)'\Omega^{-1}(y-(I-\lambda W)^{-1}X\beta) \nonumber
\end{equation}
\end{scriptsize}

note that $|\sigma^{2}\Omega|=\sigma^{2n}|\Omega|$, and that 
\begin{align}
|\Omega| &= |[(I-\lambda W)'(I-\lambda W)]^{-1}| \nonumber \\
&= |(I-\lambda W)^{-1}(I-\lambda W')^{-1}| \nonumber \\
&=|(I-\lambda W)^{-1}||(I-\lambda W')^{-1}| \nonumber \\
&=|(I-\lambda W)|^{-2} \nonumber \\
\end{align}

\end{frame}
%----------------------------------------------------------------------%
\begin{frame}[fragile]
\frametitle{Spatial Lag Model}
\framesubtitle{Maximum Likelihood Estimator}


so returning to the log likelihood we have that  the log likelihood is 

\begin{align}
l\left(\sigma^{2},\lambda,y\right)&=constant-\frac{n}{2}ln\left(\sigma^{2}\right)+ln\left(|(I-\lambda W)|\right) \nonumber \\
&-\frac{1}{2\sigma^{2}}(y-(I-\lambda W)^{-1}X\beta)'(I-\lambda W)'(I-\lambda W)(y-(I-\lambda W)^{-1}X\beta)
\end{align}

then 

\begin{align}
l\left(\sigma^{2},\lambda,y\right)&=constant-\frac{n}{2}ln\left(\sigma^{2}\right) \nonumber \\
&-\frac{1}{2\sigma^{2}}((I-\lambda W)y-X\beta)'((I-\lambda W)-X\beta) \nonumber \\
&+ln\left(|(I-\lambda W)|\right)
\end{align}


\end{frame}
%----------------------------------------------------------------------%
\begin{frame}[fragile]
\frametitle{Spatial Lag Model}
\framesubtitle{Maximum Likelihood Estimator}

\begin{itemize}
\item The determinant $|(I-\lambda W)|$ is quite complicated because in contrast to the time series, where it is a triangular matrix, here it is a full matrix. 
\item However, Ord (1975) showed that it can be expressed as a function of the eigenvalues $\omega_{i}$
\end{itemize}


\[
|(I-\lambda W)|=\Pi_{i=1}^{n}(1-\lambda\omega_{i})
\]

So the log likelihood is simplified to 

\begin{align}
l\left(\sigma^{2},\lambda,y\right)&=constant-\frac{n}{2}ln\left(\sigma^{2}\right) \nonumber \\
&-\frac{1}{2\sigma^{2}}((I-\lambda W)y-X\beta)'((I-\lambda W)-X\beta) \nonumber \\
&+\sum ln(1-\lambda\omega_{i})
\end{align}

\end{frame}
%----------------------------------------------------------------------%
\begin{frame}[fragile]
\frametitle{Spatial Lag Model}
\framesubtitle{Maximum Likelihood Estimator}

Applying  FOC, the ML estimates for $\beta$ and $\sigma^{2}$ are:
\bigskip
\[
\beta_{MLE}=(X'X)^{-1}X'(I-\lambda W)y
\]
\bigskip
\[
\sigma_{MLE}^{2}=\frac{1}{n}(y-\lambda Xy-X\beta_{MLE})'(y-\lambda Xy-X\beta_{MLE})
\]

\bigskip
\begin{itemize}
\item Conditional on $\lambda$ these estimates are simply OLS applied to the spatially filtered dependent variable and explanatory variables X.
\end{itemize}


\end{frame}
%----------------------------------------------------------------------%
\begin{frame}[fragile]
\frametitle{Spatial Lag Model}
\framesubtitle{Maximum Likelihood Estimator}
\begin{itemize}
\item  Substituting these in the log likelihood we have a concentrated log-likelihood as a nonlinear function of a single parameter $\lambda$
\end{itemize}
 
 \bigskip
\begin{align}
l\left(\lambda\right)=-\frac{n}{2}ln\left(\frac{1}{n}(e_{0}-\lambda e_{L})'(e_{0}-\lambda e_{L})\right)+\sum ln(1-\lambda\omega_{i})
\end{align}
\bigskip
\begin{itemize}
\item where $e_{0}$ are the residuals in a regression of $y$ on $X$ and
\item $e_{L}$ of a regression of $Wy$ on $X$. 
\bigskip
\item This expression can be maximized numerically to obtain the estimators for the unknown parameters $\lambda$.
\end{itemize}



\end{frame}
%----------------------------------------------------------------------%
\begin{frame}[fragile]
\frametitle{Spatial Lag Model}
\framesubtitle{Maximum Likelihood Estimator}
 The asymptotic variance follows as the inverse of the information matrix
\bigskip
\begin{scriptsize}
\begin{align}
AsyVarl\left(\lambda,\beta,\sigma^{2}\right)=\left(\begin{array}{ccc}
tr(W_{A})^{2}+tr(W_{A}'W_{A})+\frac{(W_{A}X\beta)'(W_{A}X\beta)}{\sigma^{2}} & \frac{(X'W_{A}X\beta)'}{\sigma^{2}} & \frac{tr(W_{A})'}{\sigma^{2}}\\
\frac{(X'W_{A}X\beta)'}{\sigma^{2}} & \frac{(X'X)}{\sigma^{2}} & 0\\
\frac{tr(W_{A})'}{\sigma^{2}} & 0 & \frac{n}{2\sigma^{4}} 
\end{array}\right)^{-1}
\end{align}
\end{scriptsize}

\bigskip
\begin{itemize}
\item where $W_{A}=W(I-\lambda W)^{-1}$. 
\item Note that 
\begin{itemize}
  \item the covariance between $\beta$ and $\sigma^{2}$ is zero, as in the standard regression model, 
  \item this is not the case for $\lambda$ and $\sigma^{2}.$ 
\end{itemize}

\end{itemize}


\end{frame}
%----------------------------------------------------------------------%
\subsection{Two-Stage Least Squares estimators}
%----------------------------------------------------------------------%
\begin{frame}[fragile]
\frametitle{Spatial Lag Model}
\framesubtitle{Two-Stage Least Squares estimators}

\begin{itemize}
\item An alternative to MLE we can us 2SLS to eliminate endogeneity.
\bigskip
\item Key is to identify proper instruments
\bigskip
\begin{itemize}
\item Need to be uncorrelated with the error term
\bigskip
\item Correlated with $WY$
\end{itemize}
\end{itemize}

\end{frame}
%----------------------------------------------------------------------%
\begin{frame}[fragile]
\frametitle{Spatial Lag Model}
\framesubtitle{Two-Stage Least Squares estimators}
Consider the following

\[
E(y)=(I-\lambda W)^{-1}X\beta
\]

now, since $|\lambda|<1$ we can use Neumann series property to expand
the inverse matrix as 

\[
(I-\lambda W)^{-1}=I+\lambda W+\lambda^{2}W^{2}+\lambda^{3}W^{3}+....
\]

hence

\[
E(y)=(I+\lambda W+\lambda^{2}W^{2}+\lambda^{3}W^{3}+....)X\beta
\]

\[
=X\beta+\lambda WX\beta+\lambda^{2}W^{2}X\beta+\lambda^{3}W^{3}X\beta+....
\]

so we can express $E(y)$ as a function of $X$, $WX$, $W^{2}X$,... 


\end{frame}
%----------------------------------------------------------------------%
\begin{frame}[fragile]
\frametitle{Spatial Lag Model}
\framesubtitle{Two-Stage Least Squares estimators}

We can use the first three elements of the expansion as instruments.
Let's define $H$ as the matrix with our instruments 

\[
H=[X,WX,W^{2}X]
\]

Now, 

\[
y=\lambda Wy+X\beta+u
\]

\[
=M\theta+u
\]

\medskip
where $M=[Wy,X]$ and $\theta=[\lambda,\beta]$. 

\end{frame}
%----------------------------------------------------------------------%
\begin{frame}[fragile]
\frametitle{Spatial Lag Model}
\framesubtitle{Two-Stage Least Squares estimators}


The first stage is

\[
M=H\gamma+\eta
\]

and 
\[
\hat{\gamma}=(H'H)^{-1}H'M
\]


\[
\hat{M}=H\hat{\gamma}=P_{H}M
\]

and the second stage is 

\begin{align}
y=\hat{M}\theta+u
\end{align}

and 

\begin{align}
\hat{\theta}_{2SLS}&=(\hat{M}'\hat{M})^{-1}\hat{M}'y \nonumber \\
&=(M'P_{H}M)^{-1}M'P_{H}y
\end{align}



\end{frame}
%----------------------------------------------------------------------%
\section{Interpretation of Parameters}
%----------------------------------------------------------------------%
\begin{frame}
\frametitle{Interpretation of Parameters}
\begin{itemize}
  \item Consider the following model for the $i-th$ observation
\begin{align}
y_i = \beta_0 +\beta_1 x_{i1}+\beta_1 x_{i2}+\dots+\beta_1 x_{ir}+\dots+\beta_1 x_{ik} \,\,\, i=1,\dots,n \nonumber
\end{align}

  \item Recall that in OLS we have


\begin{align}
\beta_1=\frac{\partial y_i}{\partial x_{i1}} \nonumber
\end{align}

or generically
\begin{align}
\beta_r&=\frac{\partial y_i}{\partial x_{ir}}  \,\,\,\,  \forall i=1,\dots,n\,and\, r=1,\dots,k \nonumber \\
\beta_r&=\frac{\partial y_i}{\partial x_{jr}} \,\,\,\,  \forall j\neq i\,and\, \forall r=1,\dots,k \nonumber
\end{align}

\item Interpretation is straight forward as long as we take into account units
\medskip
\item In spatial models the interpretation is less immediate and require some clarification
\end{itemize}

\end{frame}
%----------------------------------------------------------------------%
\begin{frame}
\frametitle{Interpretation of Parameters}
\begin{itemize}
\item Lets consider the case of a simple Spatial Lag model with a single regressor
\end{itemize}

\begin{align}
y_i = \alpha + \beta x_i + \lambda \sum w_{ij} y_j + \epsilon_i
\end{align}
with $|\lambda|<1$, and 

\begin{align}
\beta \neq \frac{\partial y_i}{\partial x_{i}} \nonumber
\end{align}

\begin{align}
 \frac{\partial y_i}{\partial x_{i}} = diag(I-\lambda W)^{-1}\beta \nonumber
\end{align}
\begin{itemize}
  \item The impact depends also on the parameter $\lambda$
  \item The impact is different in each location
\end{itemize}  


\end{frame}
%----------------------------------------------------------------------%
\begin{frame}
\frametitle{Interpretation of Parameters}
More generally consider 

\begin{align}
y&=\lambda Wy+X\beta+u \nonumber \\
 &=(I-\lambda W)^{-1}X\beta+(I-\lambda W)^{-1}u \nonumber
\end{align}

Then 
\begin{align}
E(y)=(I-\lambda W)^{-1}X\beta
\end{align}

we define

\begin{align}
S_r(W)=(I-\lambda W)^{-1}\beta_r
\end{align}

\end{frame}
%----------------------------------------------------------------------%
\begin{frame}
\frametitle{Interpretation of Parameters}
\begin{align}
S_r(W)=(I-\lambda W)^{-1}\beta_r
\end{align}

we can write 
\begin{align}
\left(\begin{array}{c}
E(y_{1})\\
\vdots\\
E(y_{i})\\
\vdots\\
E(y_{n})
\end{array}\right)=\sum_{r=1}^{k}\left(\begin{array}{ccccc}
S_{r}(W)_{11} & S_{r}(W)_{12} & \dots & \dots & S_{r}(W)_{1n}\\
\\
S_{r}(W)_{i1} & \dots & S_{r}(W)_{ii} & \dots & S_{r}(W)_{in}\\
\\
S_{r}(W)_{n1} & \dots & \dots & \dots & S_{r}(W)_{nn}
\end{array}\right)\left(\begin{array}{c}
x_{1r}\\
\vdots\\
\vdots\\
\vdots\\
x_{nr}
\end{array}\right) \nonumber
\end{align}

\end{frame}
%----------------------------------------------------------------------%
\begin{frame}
\frametitle{Interpretation of Parameters}
Then for the $i-th$ observation

\begin{align}
E(y_i)=\sum_{r=1}^k (S_r(W)_{i1} x_{1r} + S_r(W)_{i2}x_{2r} + \dots + S_r(W)_{ii}x_{ir} + \dots + S_r(W)_{in}x_{nr}) \nonumber
\end{align}

then

\begin{align}
\frac{\partial E(y_i)}{\partial x_{jr}}=S_r(W)_{ij}
\end{align}
and
\begin{align}
\frac{\partial E(y_i)}{\partial x_{ir}}=S_r(W)_{ii}
\end{align}

where $S_r(W)_{ij}$ denotes the $(i,j)-th$ element of the matrix $S_r(W)$
\end{frame}
%----------------------------------------------------------------------%
\begin{frame}
\frametitle{Interpretation of Parameters}
Therefore the impact of {\it each variable}  $x_r$ on $y$ can be described through the partial derivatives $\frac{\partial E(y)}{\partial x_r}$ which can be arranged in the following matrix:

\bigskip

\begin{align}
S_r(W)=\frac{\partial E(y)}{\partial x_r}=\left(\begin{array}{ccc}
\frac{\partial E(y_{1})}{\partial x_{1r}} & \dots & \frac{\partial E(y_{1})}{\partial x_{nr}}\\
\vdots & \ddots & \vdots\\
\frac{\partial E(y_{i})}{\partial x_{1r}} & \dots & \frac{\partial E(y_{i})}{\partial x_{nr}}\\ 
\vdots & \ddots & \vdots\\
\frac{\partial E(y_{n})}{\partial x_{1r}} & \dots & \frac{\partial E(y_{n})}{\partial x_{nr}}
\end{array}\right)
\end{align}

%then

%\begin{align}
% \frac{\partial E(y_i)}{\partial x_{jr}}=S_r(W)_{ij}
%\end{align}
%and

%\begin{align}
% \frac{\partial E(y_i)}{\partial x_{ir}}=S_r(W)_{ii}
%\end{align}

\end{frame}
%----------------------------------------------------------------------%
\begin{frame}
\frametitle{Interpretation of Parameters}
On this basis, LeSage and Pace (2009) suggested the following impact measures that can be calculated for each independent variable $X_i$ included in the model 

\begin{itemize}
\item {\it Average Direct Impact}: this measure refers to the impact of changes in the $i‐th$ observation of $x_r$, which we denote $x_{ir}$, on $y_i$. This is the average of all diagonal entries in $S$
\end{itemize}

\begin{align}
ADI &= \frac{tr(S_r(W))}{n} \nonumber \\
&= \frac{1}{n}\sum_{i=1}^n S_r(W)_{ii}
\end{align}


\end{frame}
%----------------------------------------------------------------------%
\begin{frame}
\frametitle{Interpretation of Parameters}

\begin{itemize}
\item {\it Average Total Impact To} an observation: this measure is related to the impact produced on one single observation $y_i$ from changing of the $r‐th$ independent variable across all other observations. For each observation this is calculated as the sum of the $i-th$ row of matrix $S$
\end{itemize}

\begin{align}
ATIT_j &= \frac{\iota'S_r(W)}{n} \nonumber \\
&= \frac{1}{n}\sum_{i=1}^n S_r(W)_{ij}
\end{align}


\end{frame}
%----------------------------------------------------------------------%
\begin{frame}
\frametitle{Interpretation of Parameters}

\begin{itemize}
\item {\it Average Total Impact From} an observation: this measure is related to the total impact on all other observations $y_i$ from changing the $r‐th$ variable in $j‐th$ observation. For each observation this is calculated as the sum of the $j-th$ column of matrix $S$
\end{itemize}

\begin{align}
ATIF_i &= \frac{1}{n}S_r(W)\iota \nonumber \\
&= \frac{\sum_{j=1}^n S_r(W)_{ij}}{n} 
\end{align}


\end{frame}
%----------------------------------------------------------------------%
\begin{frame}
\frametitle{Interpretation of Parameters}

\begin{itemize}
\item A Global measure of the average impact obtained from the two previous measures. 
\item It is simply the average of all entries of matrix S
\end{itemize}

\begin{align}
ATI &= \frac{1}{n}\iota'S_r(W) \iota = \frac{1}{n}\sum_{i=1}^n\ ATIT_i = \frac{1}{n}\sum_{j=1}^n\ ATIF_i
\end{align}

\begin{itemize}
  \item The numerical values of the summary measures for the two forms of average total impacts are equal.
  \item The ATIF relates how changes in a single observation j influences all observations.
  \item In contrast, the ATIT considers how changes in all observations influence a single observation i.
\end{itemize}


\end{frame}
%----------------------------------------------------------------------%
\begin{frame}
\frametitle{Interpretation of Parameters}

\begin{itemize}
\item {\it Average Indirect Impact} obtained as the difference between ATI and ADI
\end{itemize}

\begin{align}
AII = ATI - ADI
\end{align}

\bigskip

\begin{itemize}
\item It is simply the average of all off-diagonal entries of matrix $S_r$
\end{itemize}


\end{frame}
%----------------------------------------------------------------------%
\begin{frame}
\frametitle{Interpretation of Parameters: Example}
\begin{itemize}
  \item We have data on 20 Italian regions on GDP and unemployment. 
  \item We want to estimate the effect of GDP on Unemployment (Okun's Law)
\end{itemize}


\begin{table}[H]
\begin{tabular}{lcc} \\
[-1.8ex] \hline \hline \\[-1.8ex]
          & OLS       & Spatial Lag Model \\
 \hline \\[-1.8ex]          
Intercept & 10.971*** & 3.12275***        \\
GDP       & ‐3.326*** & ‐1.13532***       \\
$\lambda$ & -         & 0.7476***         \\
ADI       & -         & ‐1.542448         \\
AII       & -         & -2.95571           \\
ATI       & -         & ‐4.498159        \\
\hline \hline \\[-1.8ex]
\end{tabular}
\end{table}

\end{frame}
%----------------------------------------------------------------------%
\begin{frame}
\frametitle{Review \& Next Steps}
  
  \begin{itemize} 
    \item Today:
    \medskip
    \begin{itemize} 
        \item Details on Spatial Lag Model
        \medskip
        \item Interpretation
        \medskip
      \end{itemize}
  	\bigskip  

	\item  Next class: Model assessment and model selection


\bigskip  
\item Questions? Questions about software? 

\end{itemize}
\end{frame}

%----------------------------------------------------------------------%
\section{Further Readings}
%----------------------------------------------------------------------%
\begin{frame}
\frametitle{Further Readings}

\begin{itemize}

   
  \item Arbia, G. (2014). A primer for spatial econometrics with applications in R. Palgrave Macmillan. (Chapter 2 and 3)
  \medskip
  \item Anselin, Luc, \& Anil K Bera. 1998. “Spatial Dependence in Linear Regression Models with an Introduction to Spatial Econometrics.” Statistics Textbooks and Monographs 155. MARCEL DEKKER AG: 237–90.
  \medskip
  \item Anselin, L. (1982). A note on small sample properties of estimators in a first-order spatial autoregressive model. Environment and Planning A, 14(8), 1023-1030.
  \medskip
  \item Tobler, WR. 1979. “Cellular Geography.” In Philosophy in Geography, 379–86. Springer.
\end{itemize}

\end{frame}






%----------------------------------------------------------------------%
%----------------------------------------------------------------------%
\end{document}
%----------------------------------------------------------------------%
%----------------------------------------------------------------------%

